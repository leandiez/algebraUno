\begin{enunciado}{\ejercicio}
  En este ejercicio no hace falta usar inducción.
  \begin{enumerate}[label=\alph*)]
    \item Probar que $\sumatoria{k = 0}{n} \binom{n}{k}^2 = \binom{2n}{n}. \qquad \parentesis{\text{sug: } \binom{n}{k} = \binom{n}{n-k}}$.
    \item Probar que $\sumatoria{k = 0}{n} (-1)^k \binom{n}{k} = 0$.
    \item Probar que $\sumatoria{k = 0}{2n} \binom{2n}{k} = 4^n$ y deducir que $\binom{2n}{n} < 4^n$.
    \item Calcular $\sumatoria{k = 0}{2n+1} \binom{2n+1}{k}$ y deducir $\sumatoria{k=0}{n} \binom{2n+1}{k}$.
  \end{enumerate}
\end{enunciado}

\begin{enumerate}[label=\alph*)]
  \item
        Voy a mostrarlo usando un argumento combinatorio. Básicamente voy a mostrar que con las dos expresiones estamos contando lo mismo.
        Imaginemos que tenemos un conjunto de $\rosa{n}$ cantidad de mujeres y otro de $\blue{n}$ cantidad de hombres.
        La suma de este conjunto tendría $\green{2n}$ personas. Ahora, yo quiero elegir $\green{n}$ personas de ese total
        de $\green{2n}$ personas, contar eso es:
        $$
          \binom{2n}{n}.
        $$

        \bigskip

        \parrafoDestacado{
          \textit{Un modelo de juguete:}

          En una caja con 6 \textit{bolitas}, podría sacar 3 de $\binom{6}{3} = \magenta{20}$ maneras diferentes.
          Ahora propongo agarrarlas de una forma diferente:

          Pinto \rosa{3 de rosa} y \blue{3 de azul}.
          Sigue habiendo 6 \textit{bolitas} solo que pintadas, ahora voy sacando de a 3, nuevamente, pero contando así:
          $$
            \binom{\rosa{3}}{\rosa{0}} \binom{\blue{3}}{\blue{3}} +
            \binom{\rosa{3}}{\rosa{1}} \binom{\blue{3}}{\blue{2}} +
            \binom{\rosa{3}}{\rosa{2}} \binom{\blue{3}}{\blue{1}} +
            \binom{\rosa{3}}{\rosa{3}} \binom{\blue{3}}{\blue{0}} =
            1 + 9 + 9 + 1 =
            \magenta{20}
          $$
          Cada término de esa suma es contar las formas de sacar $\rosa{k}$ \textit{bolitas} rosas
          de las \rosa{3 \textit{bolitas} rosas} que hay para luego multplicar eso por la cantidad de sacar
          \textit{\blue{$(3-k)$ bolitas azules}}. Como estoy sacando $\rosa{k} + \blue{(3-k)}$ es siempre
          estar sacando 3.
        }

        \bigskip

        \textit{El modelo más pulenta:}

        Hasta ahora todo bien. Notemos que esto lo puedo decir también, es decir, es lo mismo que elegir
        $\rosa{k}$ mujeres de las $\rosa{n}$ mujeres que hay y elegir $\blue{n} - \rosa{k}$ hombres entre los \blue{n} hombres que hay.

        Notar que $\rosa{k} + (\blue{n} - \rosa{k}) = \green{n}$, así que el grupo que elijamos como combinación
        de los dos siempre va a tener $\green{n}$ personas, y va a estar elegido una parte desde $\rosa{n}$ mujeres y la otra desde $\blue{n}$ hombres.

        \medskip

        Entonces voy a sumar todas las posibles formas de elegir $\green{n}$ personas de entre $\green{2n}$ personas,
        pero agarrando siempre de $\rosa{k}$ mujeres y $\blue{n} - \rosa{k}$ hombres :
        $$
          \binom{\green{2n}}{\green{n}} =
          \ob{
            \ub{\displaystyle \binom{\rosa{n}}{\rosa{0}}\binom{\blue{n}}{\blue{n} - \rosa{0}}}
            {\substack{\text{elijo  \rosa{0} mujeres}\\
                \ytext \\
                \text{ $\blue{n}$ hombres}
              }}
            +
            \ub{\displaystyle \binom{\rosa{n}}{\rosa{1}}\binom{\blue{n}}{\blue{n} - \rosa{0}}}
            {\substack{\text{elijo  $\rosa{1}$ mujeres}\\
                \ytext \\
                \text{ $\blue{n} - \rosa{1}$ hombres}
              }}
            +
            \cdots
            +
            \ub{\displaystyle \binom{\rosa{n}}{\rosa{n-1}}\binom{\blue{n}}{\blue{n} - (\rosa{n - 1})}}
            {\substack{\text{elijo  \rosa{$n-1$} mujeres}\\
                \ytext \\
                \text{ $\blue{1}$ hombres}
              }}
            +
            \ub{\displaystyle \binom{\rosa{n}}{\rosa{n}}\binom{\blue{n}}{\blue{n} - \rosa{n}}}
            {\substack{\text{elijo  \rosa{$n$} mujeres}\\
                \ytext \\
                \text{ $\blue{0}$ hombres}
              }}
          }{\text{formas de tomar \green{$n$} personas de un total entre $\green{2n}$}}
          = \llamada1
        $$
        La sumatoria en $\llamada1$ se puede reescribir por su simetría y además por la sugerencia del enunciado como:
        $$
          \llamada1 =
          \sumatoria{\rosa{k} = 0}{n} \binom{\rosa{n}}{\rosa{k}}\binom{\blue{n}}{\blue{n} - \rosa{k}}
          \igual{\red{!}}[sug.]
          \sumatoria{\rosa{k} = 0}{\green{n}} \binom{\rosa{n}}{\rosa{k}}^2
          = \binom{\green{2n}}{\green{n}}
        $$
        Como vimos, estamos contando lo mismo con las dos expresiones, por lo tanto queda probado que son iguales.
        $$
          \cajaResultado{
            \sumatoria{k = 0}{n} \binom{n}{k}^2
            = \binom{2n}{n}
          }
        $$

  \item
        ¿Vale usar \textit{crudamente} la fórmula del binomio de Newton acá?

        \bigskip

        \textit{Si \yellow{\underline{sí}}, se puede: }
        $$
          \textstyle
          \ub{(x + y)^n = \sumatoria{k=0}{n} \binom{n}{k} x^n y^{n-k}}{\text{Binomio de Newton}}
          \Entonces{$x = 1$}[$y = -1$]
          (1 - 1)^n = \sumatoria{k=0}{n} \binom{n}{k} 1^n \cdot (-1)^{n-k}
          \sii
          \cajaResultado{
            \sumatoria{k=0}{n} (-1)^{n-k} \binom{n}{k} = 0
          }
        $$

        \bigskip

        \bigskip

        \textit{Si \yellow{\underline{no}}, se puede: }
        Con una idea de por donde va esto de sumar los números combinatorios dada su simetría:

        \textit{Caso con $n$ impar:}
        $$
          \begin{array}{rcl}
            \sumatoria{k = 0}{n} (-1)^k \binom{n}{k}
             & =                      &
            (-1)^0  \binom{n}{0} +
            (-1)^1  \binom{n}{1} +
            \cdots +
            (-1)^{\frac{n}{2}} \binom{n}{\frac{n}{2}} +
            \cdots +
            (-1)^{(n-1)} \binom{n}{n-1} +
            (-1)^n  \binom{n}{n}        \\
             & \igual{\red{!}}        &
            \binom{n}{0} -
            \binom{n}{1} +
            \cdots \magenta{\pm}
            \binom{n}{\frac{n-1}{2}} \magenta{\mp}
            \binom{n}{\frac{n+1}{2}} \magenta{\pm}
            \cdots +
            \binom{n}{n-1} -
            \binom{n}{n}                \\
             & \igual{\red{!!}}[sug.] &
            \binom{n}{0} -
            \binom{n}{1} +
            \cdots \magenta{\pm}
            \binom{n}{\frac{n-1}{2}} \magenta{\mp}
            \binom{n}{\frac{n-1}{2}} \magenta{\pm}
            \cdots +
            \binom{n}{1} -
            \binom{n}{0}
            =0
          \end{array}
        $$

        \textit{Caso con $n$ par:}
        $$
          \begin{array}{rcl}
            \sumatoria{k = 0}{n} (-1)^k \binom{n}{k}
             & =                      &
            (-1)^0  \binom{n}{0} +
            (-1)^1  \binom{n}{1} +
            \cdots +
            (-1)^{\frac{n}{2}} \binom{n}{\frac{n}{2}} +
            \cdots +
            (-1)^{(n-1)} \binom{n}{n-1} +
            (-1)^n  \binom{n}{n}        \\
             & \igual{\red{!}}        &
            \binom{n}{0} -
            \binom{n}{1} +
            \cdots +
            \binom{n}{\frac{n}{2}} -
            \cdots -
            \binom{n}{n-1} +
            \binom{n}{n}                \\
             & \igual{\red{!!}}[sug.] &
            \binom{n}{0} -
            \binom{n}{1} +
            \cdots +
            \binom{n}{\frac{n}{2}} -
            \cdots -
            \binom{n}{1} +
            \binom{n}{0}                \\
             & =                      &
            2\binom{n}{0} -
            2\binom{n}{1} +
            \cdots -
            2\binom{n}{\frac{n}{2}-1} +
            \binom{n}{\frac{n}{2}}
          \end{array}
        $$

        \red{Tengo que probar que eso me da cero... a ver si alguien por favor me saca de este quilombo}

        \hacer

  \item
        \hacer
  \item
        \hacer
\end{enumerate}

\begin{aportes}
  \item \aporte{https://github.com/sigfripro}{sigfripro \github}
  \item \aporte{\dirRepo}{naD GarRaz \github}
\end{aportes}
