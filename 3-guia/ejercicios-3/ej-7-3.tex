\begin{enunciado}{\ejercicio}
  María tiene una colección de 17 libros distintos que quiere guardar en 3 cajas:
  una roja, una amarilla y una azul.
  ¿De cuántas maneras distintas puede distribuir los libros en las cajas?
\end{enunciado}

Tenemos un conjunto de libros y otro de cajas, todos los libros van a ir a parar a al menos una caja.
Esto quiere decir que basicamente lo que queremos buscar son todas las funciones de libros a cajas. 
Que se obtiene haciendo $\#(\text{cajas})^{\#(\text{libros})} = \cajaResultado{3^{17}}$ \\
Otra forma de pensarlo es que por cada libro tenemos $3$ opciones de mandarlo a cualquiera de las cajas,
como tenemos 17 libros, esto es
\[\overbrace{3 \times 3 \times \cdots \times 3}^{17\ \text{veces}} = 3^{17}\]

\begin{aportes}
  \item \aporte{https://github.com/sigfripro}{sigfripro \github}
\end{aportes}
