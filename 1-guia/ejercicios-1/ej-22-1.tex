\def\veintidosi{
  \begin{tikzpicture}[baseline=0, >=Latex, draw=Aquamarine,scale=0.7, transform shape]

    \node[] (1) {$\bullet$};
    \node[] at (1.west) {$1$};

    \node[above right= of 1] (2) {$\bullet$};
    \node[] at (2.east) {$2$};

    \node[below right= of 2] (3) {$\bullet$};
    \node[] at (3.east) {$3$};

    \node[below= of 1] (4) {$\bullet$};
    \node[] at (4.west) {$4$};

    \node[right= of 2] (5) {$\bullet$};
    \node[] at (5.west) {$5$};

    % Universo
    \node[shape=ellipse, draw, black, fit={ (1) (2) (3) (4)}] (universo) {};
    \node[above left = 0.1em of universo] {$A$};

    % Aristas
    \draw[->, loop below] (1) to (1);
    \draw[->, loop left] (2) to (2);
    \draw[->, loop above] (3) to (3);
    \draw[->, loop above] (4) to (4);
    \draw[->, loop right] (5) to (5);

    \draw[->, bend right] (1.center) to (2);
    \draw[->, bend right] (1.center) to (3);
    \draw[->, bend right] (1.center) to (5);

    \draw[->, bend left] (2.center) to (5);
  \end{tikzpicture}
}
% fin gráficos
\begin{enunciado}{\ejercicio}
  En cada uno de los siguientes casos determinar si la relación $\relacion $ en $A$ es reflexiva, simétrica,
  antisimétrica, transitiva, de equivalencia o de orden.
  \begin{enumerate}[label=\roman*)]
    \item $A = \set{1,2,3,4,5}, \relacion = \set{(1,1), (2,2), (3,3), (4,4), (5,5), (1,2), (1,3), (2,5), (1,5)}$
    \item $A = \naturales,\, \relacion = \set{(a,b) \en \naturales \times \naturales \big/ a+b \text{ es par} }.$
    \item $A = \enteros,\, \relacion = \set{(a,b) \en \enteros \times \enteros \big/ |a| \leq |b| }.$
    \item $A = \enteros,\, \relacion$ definida por $a \relacion b \sii b \text{ es múltiplo de } a$.
    \item $A = \partes(\reales),\, \relacion$ definida por $X \relacion Y \sii X \inter \set{1,2,3} \subseteq Y \inter \set{1,2,3}$.
    \item $A = \partes(\set{n \en \naturales \talque n \leq 30})$, $\relacion$ definida por $X \relacion Y \sisolosi 2 \notin X \inter Y^c$
    \item $A = \naturales \times \naturales,\, \relacion$ definida por $(a,b) \relacion (c,d) \sisolosi bc$ es múltiplo de $ad$.
  \end{enumerate}

\end{enunciado}

Voy a estar usando cosas del \hyperlink{teoria-1:relaciones}{resumen teórico de relaciones}.

\begin{enumerate}[label=\roman*)]
  \item Haciendo un gráfico en estos ejercicios de pocos elementos sale fácil.\par

        \begin{minipage}{0.60\textwidth}
          \textit{Reflexiva:}\par
          Es reflexiva, porque hay bucles en todos los elementos de $A$.

          \textit{Simétrica:}\par
          No es simétrica, dado que existe $(1, 5)$, pero no $(5, 1)$

          \textit{Anti-Simétrica:}\par
          Es antisimétrica. No hay ningún par que tenga la vuelta, excepto los casos $x \relacion x$.

          \textit{Transitiva:}\par
          Es transitiva. La terna 1, 2, 5 es transitiva.
          La relación es R, AS y T, por lo tanto es una \textit{relación de orden}.
        \end{minipage}
        \begin{minipage}{0.3\textwidth}
          \quad  \veintidosi
        \end{minipage}

  \item Primero notemos que $a + b$ es par puede pensarse como $a + b = 2k, k \in \naturales$, esto lo digo
  para ver que como tenemos una igualdad, nos da una idea de que muy probablemente sea de equivalencia... para tener en cuenta no mas. \\
  \textit{Reflexiva:}\par 
  Hay que probar que $\forall a \in A, a \relacion a$. \\
  Elijo un elemento arbitrario $a_0$, quiero ver que $a_0 \relacion a_0$. O sea que $a_0 + a_0$ es par, o
  dicho de otro modo: $a_0 + a_0 = 2k$, usando esta ultima afirmacion, veo que $a_0 + a_0 = 2a_0$, que efectivamente
  es un numero de la forma $2k$. Por lo tanto \underline{es reflexiva}. \\
  
  \textit{Simetrica: }\par
  Que una relacion sea simetrica significa que $a \relacion b \entonces b \relacion a$. 
  Es bastante trivial la verdad porque la suma es conmutativa, pero bueno, fijamos $a = a_0$, $b = b_0$. 
  Como $a \relacion b$, entonces $a_0 + b_0 = 2k$, pero $a_0 + b_0 = b_0 + a_0$ porque la suma es conmutativa,
  por lo tanto $b \relacion a$. Por lo tanto \underline{es simetrica}. \\

  \textit{Transitividad: }\par
  Que una relacion sea transitiva significa que $a \relacion b \ytext b \relacion a \entonces a \relacion c$. 
  Para mi la forma mas sencilla y clara de hacer este tipo de pruebas es armar un sistema de ecuaciones y ver
  que sale. Tenemos: \par
  $\begin{cases}
    a_0 + b_0 = 2k \\
    b_0 + c_0 = 2k
  \end{cases}
  \flecha{lo mismo que} \quad 
  \begin{cases}
    a_0 = 2k - b_0 \\
    c_0 = 2k - b_0
  \end{cases}
  \flecha{sumamos las ecuaciones} \quad
  a_0 + c_0 = 4k - 2b_0 \quad \llamada1
  $
  \par
  $\llamada1$ se ve que toda esa expresion equivale a un numero de la forma $2k$, por lo tanto $a_0 + c_0 = 2k$.\\
  \underline{Probando la transitividad}\\

  \textit{Antisimetrica: }\par
  Que una relacion sea antisimetrica significa que $a \relacion b \ytext b \relacion a \entonces a = b$. \\
  Podemos facilmente ver un contraejemplo, $2\relacion4$ y $4\relacion2$, pero $2\neq4$. 
  Ya con el contraejemplo queda probado que no es antisimetrica, pero quiero a parte mostrar una prueba mas general para 
  el que le interese...\\
  Tenemos:
  \[
  \begin{cases}
    a_0 + b_0 = 2k\\
    b_0 + a_0 = 2k
  \end{cases}
  \]
  Vemos que por conmutatividad de la suma las dos expresiones son identicas. LLamo $h = 2k$ y tengo: \\
  $a_0 + b_0 = h \flecha \quad a_0 = h - b_0$. De aca obtengo otra manera de expresar la relacion como
  \[
  (h - b_0) \relacion b_0
  \] 
  Veo que si elijo un $h$ cualquiera (tiene que ser par), puedo elegir tambien cualquier $b_0$, y la regla de 
  antisimetria no se va a cumplir para todos los numeros.

  \item \hacer
  \item \hacer
  \item \hacer
  \item $
          A =
          \partes(\set{n \en \naturales \talque n \leq 30})$, $\relacion$ definida por $X \relacion Y \sisolosi 2 \notin X \inter Y^c
        $\par
        $$
          \begin{array}{|c|c|c|c|c|c|}
            \hline
            2 \en X     & 2 \en Y     & 2 \en Y^c   & 2 \en X^c   & 2 \notin X \inter Y^c & 2 \notin Y \inter X^c \\ \hline  \hline
            \magenta{V} & \magenta{V} & \magenta{F} & \magenta{F} & \magenta{V}           & \magenta{V}           \\
            \cyan{V}    & \cyan{F}    & \cyan{V}    & \cyan{F}    & \cyan{F}              & \cyan{V}              \\
            \cyan{F}    & \cyan{V}    & \cyan{F}    & \cyan{V}    & \cyan{V}              & \cyan{F}              \\
            \magenta{F} & \magenta{F} & \magenta{V} & \magenta{V} & \magenta{V}           & \magenta{V}           \\ \hline
          \end{array}$$.

        \textit{Reflexiva:}\par
        La relación es reflexiva ya que para que un elemento $X$ esté relacionado con sí mismo debe ocurrir
        que $X \relacion X \sisolosi 2 \notin X \inter X^c$, es decir $2 \notin \vacio$, lo cual es siempre cierto.

        \textit{Simétrica:}\par
        La relación no es simétrica. Se puede ver con la \cyan{segunda y tercera} fila de la tabla con un contraejemplo.
        $X = \set{1}$ y $Y = \set{2},\, X,Y \subseteq A$, $X \relacion Y$, pero $Y \norelacion X$,

        \textit{Anti-Simétrica:}\par
        La relación no es antisimétrica. Se puede ver con la \magenta{primera o cuarta} fila tabla con un contraejempl
        con un contraejemplo. Si $X = \set{1,2}$ e $Y = \set{2,3} \entonces X \relacion Y$ y además $Y \relacion X$
        con  $X \distinto Y$.

        \textit{Transitiva:}\par
        Es transitiva. Si bien no es lo más fácil de explicar, se puede ver en la tabla que para tener 2 relaciones
        en una terna $X, Y, Z$ no se puede llegar nunca al caso de la segunda fila de la tabla, donde se lograría que
        $X \norelacion Z$

  \item

        $A = \naturales \times \naturales,\, \relacion$ definida por $(a,b) \relacion (c,d) \sisolosi bc$ es múltiplo de $ad$.

        \bigskip

        \textit{Reflexiva:}

        $(a,b) \relacion (a,b) \sisolosi ba = k\cdot ab$ con $k=1$.
        Se concluye que $\relacion$ sí es \textit{reflexiva}.

        \textit{Simétrica:}

        Hago un contraejemplo,
        $$
          \llave{l}{
            (1,2) \relacion (3,3) \sisolosi 2 \cdot 3 \igual{def} k\cdot 1 \cdot 3 = 3 \cdot k \\
            (3,3) \relacion (1,2) \sisolosi 1 \cdot 3 \igual{def} h \cdot 2 \cdot 3 = 6 \cdot h
          }
        $$
        La relación no es simétrica, dado que no hay un $h \en \enteros$ tal que $3 = 6 \cdot h$

        \textit{Anti-Simétrica:}

        Contraejemplo:
        Tomo dos valores distintos y veo que están relacionados
        $$
          \llave{l}{
            (1,2) \relacion (2,4) \sisolosi 2 \cdot 2 \igual{def} k \cdot 1 \cdot 4 = 4 \cdot k \\
            (2,4) \relacion (1,2) \sisolosi 4 \cdot 1 \igual{def} k \cdot 2 \cdot 2 = 4 \cdot h
          }
        $$

        $(1,2) \relacion (2,4) \ytext (2,4) \relacion (1,2)$ con $(1,2) \neq (2,4)$.

        Por lo tanto la relación $\relacion$ no es \textit{antisimétrica}.

        \textit{Transitiva:}
        $$
          \llave{l}{
            (a,b) \relacion (c,d) \sisolosi bc \igual{$\llamada1$} k\cdot ad \\
            (c,d) \relacion (e,f) \sisolosi de \igual{$\llamada1$} h\cdot cf \\
            \qvq (a,b) \relacion (e,f) \sisolosi \magenta{be = k'\cdot af}   \\
            \flecha{multiplico}[M.A.M.]
            \llaves{l}
            {
              bc \igual{$\llamada1$} k\cdot ad \\
              de \igual{$\llamada1$} h\cdot cf
            } \flecha{y}[acomodo]
            be \cdot \cancel{cd} = k \cdot h \cdot af \cdot \cancel{cd} \to
            \magenta{be \igual{\Tilde} k' \cdot af}.
          }$$
        Se concluye que la relación es transitiva.

        Con esos resultados se puede decir que $\relacion$ en $A$ no es de \textit{equivalencia}
        ni de \textit{orden}.

\end{enumerate}

\begin{aportes}
  \item \aporte{\dirRepo}{naD GarRaz \github}
  \item \aporte{\neverGonnaGiveYouUp}{Magui \youtube}
  \item \aporte{https://github.com/sigfripro}{sigfripro \github}
\end{aportes}
